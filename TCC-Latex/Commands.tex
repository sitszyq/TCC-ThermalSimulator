%Commands
\newcommand{\authorname}{\textsc{Victor Ant�nio Paludetto Magri}}
\newcommand{\tctitle}{\textsc{Desenvolvimento de um Modelo T�rmico para o Acoplamento Po�o-Reservat�rio}}

% CHAPTER TITLE SETTINGS
\renewcommand{\ABNTbibliographyname}{Refer�ncias}
\renewcommand{\ABNTchaptersize}{\Large}
\renewcommand{\ABNTsectionfontsize}{\large}
\renewcommand{\ABNTsubsectionfontsize}{\normalsize}
\renewcommand{\ABNTchapterfont}{\bfseries}
\renewcommand{\ABNTsectionfont}{\bfseries \scshape}

%Symbols
\newcommand{\deriv}[2]{\dfrac{\partial}{\partial #2}\!\left( #1 \right)}
\newcommand{\derivOne}[2]{\dfrac{\partial #1}{\partial #2}}
\newcommand{\resol}{\noindent\textbf{Resolu��o.}}
\newcommand{\derivTwo}[2]{\dfrac{\partial ^2 #1}{\partial #2 ^2}}
\newcommand{\derivThree}[2]{\dfrac{\partial ^3 #1}{\partial #2 ^3}}
\newcommand{\derivFour}[2]{\dfrac{\partial ^4 #1}{\partial #2 ^4}}
\newcommand{\derivFive}[2]{\dfrac{\partial ^5 #1}{\partial #2 ^5}}
\newcommand{\derivSix}[2]{\dfrac{\partial ^6 #1}{\partial #2 ^6}}
\newcommand{\derivCross}[3]{\dfrac{\partial ^2 f}{\partial x \partial y}}
\newcommand{\fig}[1]{figura \ref{#1}}
\newcommand{\eq}[1]{equa��o \eqref{#1}}
\newcommand{\eqs}{equa��es}
\newcommand{\tab}[1]{tabela (\ref{1})}
\newcommand{\inver}[1]{\dfrac{1}{#1}}
\newcommand{\perm}{\mathbf{K}}
\renewcommand{\d}{\,\,\textrm{d}}
\newcommand{\vel}{\mathbf{v}}
\renewcommand{\div}[1]{\nabla\cdot\left(#1\right)}
\newcommand{\lap}{\nabla^2}
\newcommand{\grad}{\nabla}
\newcommand{\cellCenter}{\textit{cell center} }
\newcommand{\dint}[1]{\displaystyle\int\limits_{#1}}
\newcommand{\Int}{\displaystyle\int\limits}
\newcommand{\Sum}{\displaystyle\sum\limits}
\numberwithin{equation}{chapter}
\numberwithin{figure}{chapter}
\numberwithin{table}{chapter}